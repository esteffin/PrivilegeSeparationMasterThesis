\section{Analysis results}
We proposed a sound flow logic based analysis technique targeted at the static detection of privilege escalations attacks on Google Chrome extension, and we developed a prototype of a tool that implements our analysis.

The analysis proposed is sound and can validate real Chrome extension against privilege escalation. Indeed it gives an upper bound of the privilege that an extension leak to an attacker according to the power of such attacker. Since the analysis is sound we can truly predict if a permission is not leaked at all.

The tool that has been developed is not yet ready, but the main goal of it are almost ready. Indeed it can analyze an extension determining all possible values of each node, and from this analysis it is able to extract all possible messages sent by any content script and by the background. With this information we can perform the analysis on systems, finding the pairs privileges of the attacker, privileges leaked to it. The program unfortunately is very slow since the nature of a desugared \ljs\ code. It takes about six or eight hours to perform an analysis, but we are going to enhance its performance. This enhancement will be done starting from \cite{TAJS, TAJSDOM} and modifying our constraint solver algorithm using a faster approach like the lazy one.

\section{Future works}%(unbundling)
\label{sec:FutureWorks}
As part of our future work, we would like to enhance the performances of our actual tool and to expand its functionality in order to obtain a full implementation of the analysis.

We also want to expand this work in order to be able not only to check bundled extensions, but also to automatically unbundle bundled extensions inserting in their code stronger security checks.

%\paragraph*{Acknowledgments}
%We thank Alvise Spanò and the PLS team members for their insightful comments.
