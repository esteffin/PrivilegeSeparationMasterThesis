\documentclass[10pt,a4paper,draft]{article}
\usepackage[utf8]{inputenc}
\usepackage[english]{babel}
\usepackage{amsmath}
\usepackage{amsfonts}
\usepackage{amssymb}

\author{Enrico Steffinlongo}
\title{PhD research proposal}

\AtBeginDocument{\renewcommand{\bibname}{References}}

\begin{document}
\maketitle
\section*{Background}
In these years there has been an increment in the usage of applications that have to interact both with user sensitive data and with the untrusted outer world. This increment is led by browser applications (HTML5 apps, browser extensions,...) and by smartphone apps, where there is little control on the quality and the safety of software. To mitigate these risks, typical proposals are built on a privilege separated architecture that divides the system in different components, giving to each component only limited permissions; to preserve interaction between isolated components, the latter can communicate using a tight Message Passing Interface. Moreover, to avoid privilege escalation attacks where a component that is compromised by an attacker can obtain permissions that it did not originally have, permissions are usually given statically, before the execution of the application. According to the permission system, there is a strict punctual mapping between components and permissions in order to reduce the possible exploits of the attacker.

The message passing interface between components is often regulated by a centralized monitor and it can deliver different kinds of messages: from simple strings to complex object, with pointers and methods. The message passing interfaces constitutes a relevant attack surface against privilege-separated applications, because it may lead to privilege escalation attacks, when a compromised, or malicious component that does not have a permission can send messages asking other components to trigger security-sensitive operations.

An important example of real systems with the feature above is Android. As discussed in \cite{AndroidMPI,AndroidPermissions,AndroidPermission,AndroidPrivilegeEscalation,AndroidRedelegation} this architecture suffers various attacks like privilege escalation, since applications written by programmers that are not security experts can expose a large attack surface, being over-privileged or lacking of proper checks on the incoming messages.

Another case of privilege separated architecture which uses a simple message passing interface is the Google Chrome Extension framework. In this architecture, that extends the functionality of the browser, as shown in \cite{ChromeExtSpec,ChromeExtSpecSnd}, the choice is to separate every app in two sets of components: Backgrounds and Content Scripts. The former is the part of the application that interacts with sensible data, such as password and cookies, and with the browser core; as such, it is isolated from the page on which the extension is running, in order to prevent a malicious script in the page from being executed with high privileges. Content Scripts, instead, have no permission except the one used to send messages to the Background, and they have access to the page on which they are injected. To avoid direct delegation of privileges from the background page to some content script, valid messages include only strings: objects, functions and pointers cannot be exchanged. This choice restricts the attack surface, since an attacker cannot send functions to be executed in the privileged background page. Unfortunately it also reduces the expressiveness of the language since any object is marshaled using a JSON serializer to a string that contains only values or pair field-value. Such serialization also fails in presence of recursive objects (e.g., DOM elements) and breaks the typical prototype-based inheritance of Javascript. This reduces severely the  framework potentiality because, dealing with this limitations, developers are often obliged to do complex workaround just to perform simple tasks.

We argue that current privilege-separated architectures suffer limitations that affect both functional and non-functional requirements, together with more complexity on the component-permission mapping. For example, even if sometimes permission can be required and revocated dynamically (e.g., Chrome Extensions \cite{ChromeExtensionOnline}), the programmer often gives to each component the maximum of the permission required throughout its execution, even if a certain privilege is only rarely used.

Another crucial aspect in the design of an architecture for developing security critical software is the choice of the adopted languages. Scripting languages tend to be easier to deal with and more expressive for developing simple application, but they suffer of their weak typing discipline because it is hard to statically check them and is very complex to grant security properties. This gives to the architecture the role of mitigating such missing. On the opposite, strongly typed languages (like Java, C\#, Haskell) are less liberal because through typing they give warranties on their behavior. This lets the architecture to be more relaxed. Unfortunately such languages are more complex and verbose, so not much suitable to small applications and developers tend to avoid them.

\section*{Proposal}
In this scenario interesting researches could examine deeper the state of the art and propose frameworks to increase security for users and robustness of applications, and tools for programmers.

In particular is interesting to explore deeply the trade-off between safety, easiness and expressivity of the chosen language. As written before the language is a crucial choice for the development and for getting security properties. In this field various solution can be adopted to enhance safety without limiting too much developers like various kind of typing such soft-typing, inference or automatic annotation to make the application more robust at run-time.

Another interesting improvement can be the enhancement of expressivity and safety of the architectures proposing more sophisticated message passing interfaces, allowing for example for example exchange of pointers and functions, better dynamic acquisition of privileges, checked delegation, strong sandboxing of some component \cite{PriviSep}, centralization or distribution of the monitor of the messages and a permission policy more fine-grained.

Such analysis will be done using both static and dynamic techniques taking in particular attention formal correctness using typical approach like typing \cite{Strobe}, Flow logic \cite{FlowLogic,CarmelFlowLogicFormalization,PrincipleProgramAnalysis}, knowledge coming from the abstract interpretation field \cite{StringAbstraction,LambdaJSMightVanHorn} and model checking.

The research will start from actual solutions, their analysis and refinement through the various approaches listed before making tools prototypes able to show the effectiveness of the solution.

\bibliographystyle{plain}
\bibliography{bibliography}
\end{document}
