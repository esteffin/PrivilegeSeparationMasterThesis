\section{Abstract domains choice} % scelta dei domini astratti
\label{sec:AbstractDomChoice}

$ R_1=\rec{\vec{\widehat{str_i}:\widehat{v_i}}} \sqsubseteq \rec{\vec{\widehat{str_j}:\widehat{v_j}}}=R_2 $ sse:
\begin{enumerate}
\item $R_1$ ha meno campi di $R_2$
\item ogni campo di $R_1$ e' piu' preciso del \textbf{corrispondente} campo di $R_2$ 
\end{enumerate}

$\forall i, \exists j: \widehat{str_i} \sqsubseteq \widehat{str_j}$\\
$\forall i, \exists j: \widehat{str_i} \sqsubseteq \widehat{str_j} \Rightarrow \widehat{v_i} \sqsubseteq \widehat{v_j}$

Set:
\begin{itemize}
\item Exact
	\begin{itemize}
	\item $\exists \rightarrow Union$
	\item $\nexists \rightarrow add in prefix$
	\end{itemize}
\item Prefix
	\begin{itemize}
	\item aggiungo in $*$
	\end{itemize}
\end{itemize}

$\vat \sqsubseteq \vat'$ iff $\forall \widehat{u}_i \in \vat, \exists \widehat{u}_j \in \vat': \widehat{u}_i \sqsubseteq \widehat{u}_j $.\\
If Galois connection then \\
$\vat \sqsubseteq \vat'$ iff $\gamma(\vat) \subseteq \gamma(\vat') $\\
where $\gamma : \widehat{V} \rightarrow P(V)$ is the concretisation function.\\
$\gamma_p : \widehat{PV} \rightarrow P(V)$\\
$\gamma(\vat) = \bigcup_{\widehat{u}_i \in \vat} \gamma_p(\widehat{u}_i)$\\
\newpage
$
\begin{array}{ll}
\widehat{pre_{bool}} = \widehat{true} | \widehat{false}\\
\widehat{u_{bool}} = \{\vec{\widehat{pre_{bool}}}\}&$ with $\sqsubseteq = \subseteq\\
\widehat{pre_{int}} = \oplus | 0 | \ominus\\
\widehat{u_{int}} = \{\vec{\widehat{pre_{int}}}\}&$ with $\sqsubseteq = \subseteq\\
\widehat{pre_{string}} = s | s*\\
\widehat{u_{string}} = \{\vec{\widehat{pre_{string}}}\}&$ with $\sqsubseteq = \subseteq\\
&$ --- Giulia's spec. is more tricky than $\subseteq\\
\widehat{pre_{ref}} = r\\
\widehat{u_{ref}} = \{\vec{\widehat{pre_{ref}}}\}&$ with $\sqsubseteq = \subseteq\\
\widehat{pre_\lambda} = \lambda\\
\widehat{u_\lambda} = \{\vec{\widehat{pre_\lambda}}\}&$ with $\sqsubseteq = \subseteq\\
\widehat{pre_{rec}} = \rec{\vec{\widehat{str}_i: \vat_i}}\\
\widehat{u_{rec}} = \widehat{pre_{rec}}&$ with $ \sqsubseteq = \widehat{u_{rec}}_\sqsubseteq\\
\vat = (\widehat{u_{bool}}, \widehat{u_{int}}, \widehat{u_{string}}, \widehat{u_{ref}}, \widehat{u_{\lambda}}, \widehat{u_{rec}}, \{\widehat{Null}\}, \{\widehat{Undef}\})\\
&$ with $ \vat \sqsubseteq \vat' $ iff $ \\
&\widehat{u_{bool}} \sqsubseteq \widehat{u_{bool}}' \wedge\\
&\widehat{u_{int}} \sqsubseteq \widehat{u_{int}}' \wedge\\
&\widehat{u_{string}} \sqsubseteq \widehat{u_{string}}' \wedge\\
&\widehat{u_{ref}} \sqsubseteq \widehat{u_{ref}}' \wedge\\
&\widehat{u_{\lambda}} \sqsubseteq \widehat{u_{\lambda}}' \wedge\\
&\widehat{u_{rec}} \sqsubseteq \widehat{u_{rec}}' \wedge\\
&\widehat{Null} \not\in \vat' \vee \widehat{Null} \in \vat \wedge \widehat{Null} \in \vat' \wedge\\
&\widehat{Undef} \not\in \vat' \vee \widehat{Undef} \in \vat \wedge \widehat{Undef} \in \vat'\\
\end{array}
$


\section{Abstract operations} %- specifica delle operazioni astratte
\label{sec:AbstractOp}

\section{Requirements verification} % verifica delle condizioni (anche semi-formale)
\label{sec:RequirementVerif}